\subsection{Arduino}
La partie matérielle concernant l’Arduino est déjà largement détaillée dans le point 2.1.2 Architecture matériel – Arduino. Dès lors, nous allons ici principalement nous attarder sur la partie logicielle et nous allons aussi expliquer ce qu’apporte l’Arduino à notre solution.

\subsubsection{But de l'Arduino}
Dans un premier temps, notre réflexion s’est portée sur la possibilité pour une ville d’ouvrir ou de fermer des quartiers aux véhicules. Ainsi, nous avons souhaité apporter une solution modulaire à ce problème.\\

Généralement, une ville dispose de rues qui deviennent piétonnes lors de certaines occasions. Par exemple, la ville de Namur rend la rue de fer (et d’autres) piétonne le samedi matin afin de pouvoir proposer un marché aux riverains. De plus, lors de certaines occasions (fêtes de Wallonie, etc.) la ville rend certaines rues piétonnes. Nous constatons donc ici que notre système doit être flexible et évoluer avec le temps. Il doit permettre de facilement ajouter des rues/zones piétonnes. Une smart city étant amenée à grandir et à évoluer, il nous fallait une solution évolutive.\\

Nous avons donc choisi de considérer l’Arduino comme un objet connecté indépendant que nous pouvons configurer et ajouter à une infrastructure déjà existante. Dès lors, le code a été conçu pour se modifier très facilement dans le but d’ajouter des zones qu’on peut ouvrir ou fermer.

\subsubsection{Gestion des zones}
Dans notre maquette, nous avons défini 9 zones que nous pouvons activer (LEDs allumées) ou désactiver (LEDs fermées). Nous avons donc les zones suivantes :

\begin{table}[H]
\centering
\captionsetup{width=\textwidth}
{\renewcommand{\arraystretch}{1.5}
    \begin{tabular}{| l | l | l | l |}
    \hline
    \textbf{Élément} & \textbf{Pin} & \textbf{Utilisation} & \textbf{Identifiant}\\
    \hline
    L1 et L2 & 1 & Zone Nord-Ouest & NW\\
    \hline
    L3 et L4 & 2 & Zone Nord-Est & NE\\
    \hline
    L5 et L6 & 6 & Zone Sud-Est & SE\\
    \hline
    L7 et L8 & 5 & Zone Sud-Ouest & SO\\
    \hline
    L9 et L11 & 4 & Zone Sud & S\\
    \hline
    L10 et L12 & 3 & Zone Nord & N\\
    \hline
    L15 et L16 & 7 & Voies prioritaires (Bus) & BUS\\
    \hline
    L13 & 0 & Déviation Est & MainRoadEast\\
    \hline
    L14 & A6 & Déviation Ouest & MainRoadWest\\
    \hline
    \end{tabular}}
    \caption{Zones gérées par l'Arduino}
\end{table}

En lisant le code de l’Arduino, nous pouvons constater que ces zones sont définies dans les premières lignes de notre code Arduino :

\begin{lstlisting}[language=C, stepnumber=5, firstnumber=1, numberfirstline=true]
//Zones configuration
char* zones[9] = {"NW", "NE", "N", "SW", "SE", "S", "BUS", "MainRoadWest", "MainRoadEast"}; //Zone managed by this arduino

//Pin to open foreach zone
int pinLed[9][3] = {
  {1, -1, -1},    //NW
  {2, -1, -1},    //NE
  {1, 2, 3},      //N
  {5, -1, -1},    //SW
  {6, -1, -1},    //SE
  {4, 5, 6},      //S
  {7, -1, -1},    //BUS
  {A6, -1, -1},   //MainRoadWest
  {0, -1, -1},    //MainRoadEast
};
\end{lstlisting}

De cette manière, nous pouvons facilement changer les zones gérées par cet Arduino. Il suffit d’ajouter ou supprimer un élément de la liste \emph{zones}. De plus, nous constatons que la variable \emph{pinLed} permet de faire correspondre à chaque zone définie dans la variable zones les LEDs à allumer/éteindre. Par exemple pour allumer la zone nord il nous suffit de trouver l’identifiant de cette zone (N) dans la variable \emph{zones}. Il se situe à l’indice 2. À présent, pour connaître les LEDs à allumer lorsque la zone nord est activée, il nous suffit d’aller voir ce que contient l’indice 2 dans la variable \emph{pinLed}. Dans notre cas, nous devrons allumer les LEDs des pins 1, 2 et 3. \\

C’est exactement ce principe que notre code utilise. Ainsi il suffit au programmeur de modifier ces deux variables pour gérer de nouvelles zones.\\

Remarques :
\begin{itemize}
\item Nous considérons bien sûr que \texttt{zones.length == pinLed.length}
\item \texttt{-1} permet de ne pas désigner de LED.\\
\end{itemize}

À présent, il nous suffit de déployer le code existant sur un autre Arduino en ayant préalablement défini les zones qu’il gérait grâce à ces deux variables et nous pourrons directement gérer d’autres zones. Dès lors, nous avons rendu notre code flexible et l’exigence que nous nous étions fixée quant à l’évolutivité d’une smartcity est respectée.

\subsubsection{Recevoir et envoyer des informations}
Maintenant que nous avons vu de manière synthétique comment notre Arduino gérait les zones qu’on lui avait définies, nous allons détailler un peu plus l’implémentation de notre solution. \\

Dans un premier temps, il est important de savoir que toutes les configurations sont définies dans les premières lignes du code de l’Arduino. À l’instar des zones expliquées précédemment, les informations de connexion au wifi, les différents points API utilisés par l’Arduino et les autres configurations sont définis avant la méthode setup de notre code. De cette manière, nous facilitons grandement la maintenabilité de notre code. \\

Lorsque l’Arduino exécute son code, il lance la méthode setup une fois. Dans cette méthode, nous allons initialiser notre Arduino. Ainsi, nous allons effectuer les actions suivantes :

\begin{enumerate}
\item \emph{initSerial} : cette fonction va initialiser le serial afin de pouvoir logger des informations via le port USB.
\item \emph{initLCD} : cette fonction va initialiser l’écran LCD lié à notre Arduino
\item \emph{initLed} : cette fonction va activer toutes les LEDs gérées par cet Arduino. Elle va les laisser allumés 5 secondes (afin de pouvoir vérifier leurs bons fonctionnements) ensuite les mettre dans leur état de base, à savoir éteint.
\item \emph{connectToWifi} : cette fonction va connecter notre Arduino à la borne wifi configurée précédemment si celui-ci n’est déjà pas connecté.
\item \emph{printWiFiStatus} : cette fonction nous permet simplement d’écrire sur le serial les informations de connexion du wifi.\\
\end{enumerate}

Une fois la méthode setup terminée, l’Arduino exécutera en boucle la méthode loop. Dans celle-ci nous avons le code métier de notre Arduino. Dans le cadre de notre projet, l’Arduino gérait non seulement les zones, mais aussi le capteur de température et d’humidité. Aussi, un écran LCD était configuré afin d’afficher les informations du capteur. La méthode Loop se compose de la manière suivante :
\begin{enumerate}
\item updateTemperatureAndHumidity : cette fonction met à jour l’humidité et la température sur base de notre capteur et affiche les informations sur l’écran LCD.
\item Mise à jour de l’état des zones : l’Arduino se connecte à l’hôte hébergeant l’API grâce à la méthode \emph{connect()}. Il fait une requête GET sur le point api donnant la liste des zones à allumer. Ensuite, la méthode \emph{readReponseContentForZone} lie le résultat de la requête et ouvre les zones concernées.
\item Mise à jour de l’humidité : l’Arduino se connecte à l’hôte hébergeant l’API grâce à la méthode \emph{connect()}. Il fait une requête POST sur le point api permettant d’ajouter une nouvelle valeur pour un senseur en envoyant les données correspondant à l’humidité.
\item Mise à jour de la température : l’Arduino se connecte à l’hôte hébergeant l’API grâce à la méthode \emph{connect()}. Il fait une requête POST sur le point api permettant d’ajouter une nouvelle valeur pour un senseur en envoyant les données correspondant à la température.
\item L’Arduino attend un délai de 3 secondes et recommence.
\end{enumerate}

\subsubsection{Méthodes et fonctions}
Voici la liste de chaque méthode/fonction définie dans le code Arduino et de leur description :
\begin{itemize}
\item \emph{Setup} : Méthode s’exécutant lors du démarrage de l’Arduino.
\item \emph{Loop} : Méthode s’exécutant en boucle et contant le code métier de l’Arduino.
\item \emph{initSerial} :  Permet d’initialiser le port Serial de l’Arduino.
\item \emph{initLCD} :   Permet d’initialiser l’écran LCD.
\item \emph{connectToWifi} : Permet de se connecter au wifi défini par les variables globales \emph{ssid} et \emph{pass}.
\item \emph{printWiFiStatus} : Imprime sur le port Serial les informations concernant le wifi.
\item \emph{connect} : Permet d’ouvrir une connexion HTTP à un hôte via un port défini  .
\item \emph{getRequest} : Permet d’effectuer une requête GET sur un une ressource d’un hôte.
\item \emph{postTemperature} : Permet d’exécuter la requête POST relative la température.
\item \emph{postHumidity} : Permet d’exécuter la requête POST relative l’humidité.
\item \emph{postRequest} : Permet d’exécuter une requête POST.
\item \emph{skipResponseHeaders} : Permet de passer les headers d’une réponse HTTP.
\item \emph{readReponseContentForZone} : Lie la réponse de la requête GET concernant les zones et ouvre/ferme les zones en fonction de cette réponse.
\item \emph{disconnect} : Permet de fermer une connexion HTTP.
\item \emph{updateTemperatureAndHumidity} : Permet de mettre à jour les variables globales humidity et temperature en fonction des triggers définis. Cette fonction affiche aussi la température et l’humidité sur l’écran LCD.
\end{itemize}
