\subsection{3 tâches}
\begin{itemize}
    \item \textbf{Passer des zones routières en zones piétonnes et inversement :}\\
    Nous entendons par là la possibilité de rendre une partie de la ville (ou la ville dans son intégralité) piétonne. Cette mesure pourrait être de courte durée ou prolongée à long terme. Afin de réaliser cette première tâche, aucun capteur n’est réellement nécessaire, seuls des feux ainsi que des écrans d’affichage seraient nécessaires. Ces derniers permettraient de dévier les véhicules pour les faire contourner les zones devenues piétonnes et ainsi bloquer l’accès à celles-ci.
    \item \textbf{Prioritiser certains véhicules :}\\
    Cette deuxième tâche a pour but de privatiser certaines voies aux transports en commun afin que ceux-ci respectent le plus possible leurs horaires annoncés. Actuellement la plupart de ces voies sont réservées en permanence ce qui n’est pas une solution adéquate. En effet, les transports en commun circulent à des fréquences différentes en fonction des périodes de la journée et de l’année (ex. : vacances scolaires), il n’est donc pas nécessaire de constamment privatiser ces bandes. Ces dernières pourraient donc être partagées avec les usagers quotidiens afin de ne pas surcharger le réseau lorsque cela n’est pas nécessaire. Nous aurons désormais besoin de capteurs, notamment des capteurs de présence de véhicules afin de vérifier si le réseau est surchargé ou non. Des périphériques d’affichages seront également utiles afin de signaler aux usagers s’ils peuvent ou non circuler sur les bandes de bus.
    \item \textbf{Adapter la circulation en fonction d'événements ou de l’environnement :}\\
    Cette troisième et dernière tâche est la plus complexe. En effet, elle permet d’adapter le trafic en fonction de plusieurs critères tels que des événements (prévus : concerts, événements sportifs, etc. et non prévus : accidents de la route, etc.), la météo, le taux de pollution, etc. Plusieurs capteurs seront dès lors nécessaires, notamment des capteurs de température au sol, des capteurs de pollution, des capteurs de présence. De plus, afin d’être au courant des événements imprévus tels qu’un accident de la route, une interface utilisateur serait nécessaire (application mobile ou web). En plus de ces trois tâches principales, il va de soi que notre système gérera l’intégralité des feux de signalisation de la ville, ceci afin d’assurer un fonctionnement “classique” de ces feux (ex. : piétons souhaitant traverser).
\end{itemize}