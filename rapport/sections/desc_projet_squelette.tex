\section{Projet squelette}
Au début du projet, il nous a été demandé de respecter un certain nombre de critères bien particuliers. Voici comment nous les avons intégrés au projet.\\

Tout d’abord, en ce qui concerne les Phidgets, nous devions utiliser au moins une Phidget board. Cela était indispensable, notre projet ne pouvant pas se passer de l’Interface Kit. Nous avions également comme consigne de récupérer l’information provenant d’au moins deux capteurs. Le système actuel récupère et utilise les informations de deux capteurs de distance IR (sur l’axe principal, pour détecter les voitures), deux capteurs de luminosité (sur l’axe auxiliaire, pour détecter les voitures. Nous n’avions pas assez de capteurs de distance IR pour les deux axes), d’un capteur de luminosité, d’un capteur d’humidité, ainsi que d’un lecteur de tag RFID. Ce dernier, combiné avec le moteur, est utilisé comme ouverture de parking. La barrière s’ouvre si l’utilisateur a payé son droit d’entrée au parking.\\

Ensuite, il était demandé de créer une couche Scala qui traiterait les données récupérées par les capteurs. Nous avons décidé d’écrire entièrement le code en Scala. En effet, tout notre système fonctionnant sous la forme de Threads interdépendants qui doivent communiquer entre eux, il nous a semblé évident d’utiliser les Acteurs de la librairie Akka pour gérer ces Threads. Il sera expliqué plus précisément comment fonctionne nos acteurs plus loin dans ce rapport (cf. \ref{systeme-acteurs}). Un autre point important était qu’il fallait agréger l’information de deux capteurs et en déduire une information particulière. Nous avons donc récupéré la température ainsi que l’humidité, et pouvons en déduire l’état de la route afin d’ajuster la vitesse. Bien que les limites soient très facilement modifiables, pour le moment la vitesse de base est 50 km/h dans toute la ville. Lorsque la température descend en dessous des 3 degrés, et que l’humidité dépasse les 70\%, la vitesse passe à 30 km/h, parce que le risque de verglas est important. Nous proposons également un web service API, qui permet de récupérer les informations de notre base de données, mais également d’effectuer des modifications, telles qu’ouvrir une zone particulière de la ville, ou ajouter une nouvelle zone. Il en va de même pour les horaires de bus, qui sont modifiables et supprimables depuis l’API.\\